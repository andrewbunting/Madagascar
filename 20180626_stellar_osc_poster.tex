%%%%%%%%%%%%%%%%%%%%%%%%%%%%%%%%%%%%%%%%%
% a0poster Portrait Poster
% LaTeX Template
% Version 1.0 (22/06/13)
%
% The a0poster class was created by:
% Gerlinde Kettl and Matthias Weiser (tex@kettl.de)
% 
% This template has been downloaded from:
% http://www.LaTeXTemplates.com
%
% License:
% CC BY-NC-SA 3.0 (http://creativecommons.org/licenses/by-nc-sa/3.0/)
%
%%%%%%%%%%%%%%%%%%%%%%%%%%%%%%%%%%%%%%%%%

%----------------------------------------------------------------------------------------
%	PACKAGES AND OTHER DOCUMENT CONFIGURATIONS
%----------------------------------------------------------------------------------------

\documentclass[a0,portrait]{a0poster}

\usepackage{multicol} % This is so we can have multiple columns of text side-by-side
\columnsep=100pt % This is the amount of white space between the columns in the poster
\columnseprule=3pt % This is the thickness of the black line between the columns in the poster

\usepackage[svgnames]{xcolor} % Specify colors by their 'svgnames', for a full list of all colors available see here: http://www.latextemplates.com/svgnames-colors

\usepackage{times} % Use the times font
%\usepackage{palatino} % Uncomment to use the Palatino font

\usepackage{graphicx} % Required for including images
\graphicspath{{figures/}} % Location of the graphics files
\usepackage{booktabs} % Top and bottom rules for table
\usepackage[font=small,labelfont=bf]{caption} % Required for specifying captions to tables and figures
\usepackage{amsfonts, amsmath, amsthm, amssymb} % For math fonts, symbols and environments
\usepackage{wrapfig} % Allows wrapping text around tables and figures
\usepackage{xcolor}
\usepackage{tcolorbox}



\begin{document}

%----------------------------------------------------------------------------------------
%	POSTER HEADER 
%----------------------------------------------------------------------------------------

% The header is divided into two boxes:
% The first is 75% wide and houses the title, subtitle, names, university/organization and contact information
% The second is 25% wide and houses a logo for your university/organization or a photo of you
% The widths of these boxes can be easily edited to accommodate your content as you see fit

\begin{minipage}[b]{0.85\linewidth}
\veryHuge \color{NavyBlue} \textbf{Stellar oscillations induced by a planetary companion} \color{Black}\\ % Title
\Huge\textit{Going off on a tangent}\\[1cm] % Subtitle
\huge \textbf{Andrew Bunting\textsuperscript{1}, John Papaloizou\textsuperscript{2} \& Caroline Terquem\textsuperscript{1}}\\[0.5cm] % Author(s)
\Large \textsuperscript{1}Physics Department, University of Oxford; \textsuperscript{2}DAMPT, University of Cambridge\\[0.4cm] % University/organization
\Large andrew.bunting@physics.ox.ac.uk\\
\end{minipage}
%
\begin{minipage}[b]{0.15\linewidth}
\includegraphics[width=0.8\linewidth]{ox_logo.png}\\
\end{minipage}

\vspace{0.5cm} % A bit of extra whitespace between the header and poster content

%----------------------------------------------------------------------------------------

\begin{multicols}{2} % This is how many columns your poster will be broken into, a portrait poster is generally split into 2 columns

%----------------------------------------------------------------------------------------
%	ABSTRACT
%----------------------------------------------------------------------------------------

\color{black} % Navy color for the abstract


\begin{tcolorbox}[colframe=black,colback=blue!10!white]

\vspace{0.5cm}

\section*{Abstract}
\Large
The gravitational potential from a planet orbiting a star causes a regular perturbation which results in oscillations in the star. These tidally induced oscillations cause changes in displacement and flux. In solving the non-adiabatic oscillation equations we found that the tangential displacement at the surface is $\sim 10^{3} - 10^{4}$ times larger than the non-adiabatic solution. This result agrees well with analytical expressions in the limit of vanishing pressure, which is approximately found at the surface. Modelling these oscillations could be used to characterise exoplanetary orbits, including determining planetary masses, and has potential for planetary detection in its own right.
\normalsize

\vspace{0.5cm}

\end{tcolorbox}

%----------------------------------------------------------------------------------------
%	INTRODUCTION
%----------------------------------------------------------------------------------------

\color{Black} % SaddleBrown color for the introduction

\begin{tcolorbox}[colframe=black,colback=blue!10!white]

\vspace{0.5cm}

\section*{Introduction}

Both the Radial Velocity (RV) and transit methods for detecting exoplanets have been successful, but both are limited -- characterising the planetary system's mass is a particular difficulty due to degeneracy with the inclination or the density, respectively. Understanding other ways in which the planet and the star interact could break these degeneracies.

Given their mass, size and proximity to their host star, hot Jupiters have a stronger interaction with their host star than other planets. The first-order effect of their presence is what is used to detect them in both the RV and transit method -- the motion of the star around the common centre of mass, and the light blocked by their presence respectively. The tidal potential due to the planet is a second-order effect which changes the amplitude and wavelength of the star's light.

Whilst modelling this interaction for the non-adiabatic case has been undertaken before \textsuperscript{\cite{Pfahl2008}}, a detailed analysis of the tangential displacement has been lacking, or has otherwise been done assuming that the change due to non-adiabaticity is small \textsuperscript{\cite{Terquem1998}}.

This work particularly focusses upon the behaviour at the very surface, where non-adiabatic effects are prominent. Comparison between the modelled behaviour and analytical results under the conditions present at the very surface show good agreement.

\vspace{0.5cm}

\end{tcolorbox}


%----------------------------------------------------------------------------------------
%	METHOD
%----------------------------------------------------------------------------------------

\color{DarkSlateGray} % DarkSlateGray color for the rest of the content

\begin{tcolorbox}[colframe=black,colback=blue!10!white]

\vspace{0.5cm}

\section*{Method}

%------------------------------------------------

\subsection*{Numerical method}

To model the oscillations, the linear non-adiabatic stellar oscillation equations were solved in the case that the star is perturbed by a regular tidal potential, due to the planet. The variables directly solved for are: $\xi_{r}$, the radial displacement; $F_{r}'$, the perturbation to the radial radiative flux; $p'$, the perturbation to the pressure; and $T'$, the perturbation to the temperature.

The equations solved are:

\small
\begin{equation} \label{eq:cont_osc}
\frac{1}{r^{2}} \frac{\partial}{\partial r} ( r^{2} \rho_{0} \xi_{r} )  
+ \left( \frac{\rho_{0}}{\chi_{\rho} p_{0}} - \frac{l (l+1)}{m^{2} \omega^{2} r^{2}} \right) p'
- \frac{\rho_{0}}{T_{0}} \frac{\chi_{T}}{\chi_{\rho}} T'
=
\frac{l (l+1)}{m^{2} \omega^{2} r^{2}} \rho_{0} \Phi_{P}
\end{equation}
\\
\begin{equation} \label{eq:ent_osc}
\left( i \rho_{0} m \omega c_{p}  + \frac{l (l+1)}{r^{2}} K_{0} \right) T'
- \left( i m \omega c_{p} \nabla_{ad} \rho_{0} T_{0}  \right) \frac{p'}{p_{0}}
+ i m \omega \rho_{0} T_{0} \frac{\partial s_{0}}{\partial r} \xi_{r}
+ \frac{1}{r^{2}} \frac{\partial}{\partial r} ( r^{2} F_{r}')
=
0
\end{equation}
\\
\begin{multline} \label{eq:flux_osc}
 \frac{F_{r}'}{K_{0}}
- \left( \frac{\partial}{\partial r} - \frac{1}{T_{0}} \frac{\partial T_{0}}{\partial r} \left[ -3 + \frac{1}{\kappa_{0}} \left( \frac{\partial \kappa}{\partial \ln T} \right)_{\rho} - \frac{\chi_{T}}{\chi_{\rho}} \left( 1 + \frac{1}{\kappa_{0}} \left( \frac{\partial \kappa}{\partial \ln \rho} \right)_{T} \right) \right] \right) T' 
- \frac{\partial T_{0}}{\partial r} \frac{1}{p_{0} \chi_{\rho}} \left( 1 + \frac{1}{\kappa_{0}} \left( \frac{\partial \kappa}{\partial \ln \rho} \right)_{T} \right) p'
=
0
\end{multline}
\\
\begin{equation} \label{eq:mom_osc}
- m^{2} \omega^{2} \rho_{0} \xi_{r} 
+ \left( \frac{\partial}{\partial r} + \frac{\rho_{0}}{\chi_{\rho} p_{0}} \frac{\partial \Phi_{0}}{\partial r} \right) p'
-  \frac{\partial \Phi_{0}}{\partial r} \frac{\rho_{0}}{T_{0}} \frac{\chi_{T}}{\chi_{\rho}} T'
=
- \rho_{0} \frac{\partial \Phi_{P}}{\partial r}
\end{equation}
\\
\normalsize
which correspond to the continuity equation, entropy equation, radiative diffusion equation, and the momentum equation respectively.


The boundary conditions are split, two apply at the centre, and two at the surface. At the centre $\xi_{r} = 0$ and $F_{r}' = 0$ which ensure continuity. At the surface $\Delta P = 0$, ensuring that the pressure at the perturbed surface is unchanged, and $\left( 4 \frac{\Delta T}{T_{0}} - \frac{\Delta F_{r}}{F_{r_{0}}} \right) = 0$, ensuring that the star remains a blackbody emitter.

The Henyey method \textsuperscript{\cite{Henyey1964}} was used to solve these equations, using a solar-type model produced using MESA \textsuperscript{\cite{Paxton2018}} as the equilibrium background star which we perturbed.

%------------------------------------------------

\subsection*{Analytical solution}

For the region at the surface, where $\Delta P \approx 0$, analytical expressions for the relationships between certain variables can be calculated by making use of this.

By taking the linearised equation of motion, and introducing $\vec{\xi}_{h} = r \vec{\nabla}_{\perp} V$, we arrive at
\small
\begin{align}
- \omega^{2} \rho \xi_{r} & = - \frac{\partial \Delta P}{\partial r} - \frac{l (l+1) \rho g V}{r} - \xi_{r} r^{2} \rho \frac{\partial \left( g / r^{2} \right) }{\partial r} - \rho \frac{\partial \Phi'}{\partial r}
\\
- \omega^{2} \rho V & = - \frac{\Delta P}{r}  -  \frac{g \rho \xi_{r}}{r} - \frac{\rho \Phi'}{r}.
\end{align}
\normalsize
Rearranging these, and taking $\Delta P = 0$, we get to the expression
\small
\begin{equation}
\label{eq:V_over_xi_r}
\frac{V}{\xi_{r}} = \frac{ \frac{g}{r \omega^{2}} \left( - \frac{\partial \Delta P}{\partial r} + \rho \Phi' \left( -\frac{2}{r} + \frac{r^{2}}{g} \frac{\partial \left( g/r^{2} \right)}{\partial r} \right) \right)  }{\left( - \frac{\partial \Delta P}{\partial r} - 2 \rho \Phi' \left( \frac{1}{r} + \frac{3 g}{r^{2} \omega^{2}} \right) \right) }.
\end{equation}
\normalsize

\vspace{0.5cm}

\end{tcolorbox}



%----------------------------------------------------------------------------------------
%	RESULTS 
%----------------------------------------------------------------------------------------

\begin{tcolorbox}[colframe=black,colback=blue!10!white]

\vspace{0.5cm}

\section*{Results}

The behaviour in the last $0.1\%$ of the stellar radius differs greatly between the adiabatic and non-adiabatic cases. The inclusion of non-adiabaticity leads to the radial displacement, $\xi_{r}$, being suppressed relative to the equilibrium value (by a factor of $10$), and the tangential displacement, $\xi_{h}$, is amplified (by a factor of $\sim 10^{2} - 10^{3}$).

Both $\xi_{r}$ and $\xi_{h}$ become negative in the surface region, leading to a phase shift of $90^{\circ}$ between the planetary motion and the surface response of the star. On top of this, the imaginary parts of $\xi_{r}$ is of similar magnitude to the real part, leading to an additional phase shift of $48^{\circ}$. The tangential displacement is dominated by its real part, leading to an additional phase shift of only $5^{\circ}$.

\begin{center}\vspace{1cm}
\includegraphics[width=0.95\linewidth]{poster_displacements}
\captionof{figure}{\color{Black} A close-up of the surface of a solar-type star, showing the logarithms of the real parts of radial and tangential displacements ($\xi_{r}$ and $\xi_{h}$, respectively) for both the non-adiabatic and the adiabatic case. In the adiabatic case $\xi_{r}$ and $\xi_{h}$ are approximately similar at the surface. Taking non-adiabatic effects into account suppresses $\xi_{r}$, and greatly amplifies $\xi_{h}$ by a factor of $\sim 10^{2} - 10^{3}$.}
\end{center}\vspace{1cm}

The rough analytical estimate for the ratio $\frac{\xi_{h}}{\xi_{r}} \sim \frac{r}{H_{p}}$, where $H_{p}$ is the pressure scale height gives a surface value of $5000$, which is definitely the right order of magnitude. Using the more involved expression of equation \ref{eq:V_over_xi_r}, we can get a more precise value, of $1300 \pm 100$, compared to the model's value of $1280 \pm 5$, where the errors are due to numerical scatter in the output.

\begin{center}\vspace{1.0cm}
\includegraphics[width=0.40\linewidth]{radial_map_complex_crop_transp}
~
~
~
~
~
~
\includegraphics[width=0.40\linewidth]{tangential_map_complex_crop_transp}
\captionof{figure}{\color{Black} A heat map showing the distribution of radial displacement ($\xi_{r}$, left, corresponding to $\sin^{2}(\theta) \cos(2 \phi + \phi_{r})$, where $\phi_{r}$ is the argument of $\xi_{r}$ at the surface) and tangential displacement ($\xi_{h}$, right, corresponding to $\sin(\theta) \cos(\theta) \cos(2 \phi)$). Blue and red correspond to the large amplitudes of opposite phase. The radial displacement has a phase lag of $47^{\circ}$, whereas the tangential displacement has a phase lag of $5^{\circ}$. The ratio of amplitudes $\frac{\xi_{h}}{\xi_{r}} \approx 10^{3}$.}
\label{fig:maps}
\end{center}\vspace{1.0cm}


Due to the spatial distribution of the tangential displacement (shown in figure \ref{fig:maps}), it averages out over the disk. Therefore it does not contribute to a change in overall brightness or to the surface-averaged radial velocity. However, it would lead to a time-dependent spectroscopic signal in an oscillatory contribution to line-broadening. If this signal is detected, it could be possible to infer the mass and inclination of the planet.

\vspace{0.5cm}

\end{tcolorbox}

%----------------------------------------------------------------------------------------
%	CONCLUSIONS
%----------------------------------------------------------------------------------------

\color{Black}

\begin{tcolorbox}[colframe=black,colback=blue!10!white]

\vspace{0.5cm}

\section*{Conclusions}

\begin{itemize}
\item Non-adiabatic effects have a strong impact upon the behaviour of oscillations at the surface
\item Tangential displacement is increased by a factor of $\sim 10^{2} - 10^{3}$ compared to the equilibrium tide
\item Analytical expressions at the surface agree well with the results of the numerical model
\item This behaviour could be utilised to characterise planetary systems, or potentially detect planets
\end{itemize}
\vspace{0.5cm}

\end{tcolorbox}

\color{Black} % Set the color back to DarkSlateGray for the rest of the content

%----------------------------------------------------------------------------------------
%	FORTHCOMING RESEARCH
%----------------------------------------------------------------------------------------
\begin{tcolorbox}[colframe=black,colback=blue!10!white]

\section*{Future work}
\vspace{-0.5cm}
The next step is to calculate the photometric and spectroscopic effects of these oscillations (both in terms of disk-averaged RV signal and time-dependent line broadening) in order to look for such signal around stars with known hot Jupiters in order to test the feasibility of this as a method.

\end{tcolorbox}

 %----------------------------------------------------------------------------------------
%	REFERENCES
%----------------------------------------------------------------------------------------

\begin{tcolorbox}[colframe=black,colback=blue!10!white]
\small
%\nocite{*} % Print all references regardless of whether they were cited in the poster or not
\bibliographystyle{plain} % Plain referencing style
\bibliography{library} % Use the example bibliography file sample.bib
\normalsize

\end{tcolorbox}
%----------------------------------------------------------------------------------------
%	ACKNOWLEDGEMENTS
%----------------------------------------------------------------------------------------
\begin{tcolorbox}[colframe=black,colback=blue!10!white]

\section*{Acknowledgements}

I would like to acknowledge people for lots of things because this is the acknowledgements section.

\end{tcolorbox}

%----------------------------------------------------------------------------------------

\end{multicols}
\end{document}